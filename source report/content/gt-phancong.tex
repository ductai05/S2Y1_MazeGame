\section{Giới thiệu}

\paragraph{}{\textbf{Đây là bài báo cáo cho Đồ án Trò chơi Mê cung, môn Thực hành - Lập trình cho Trí tuệ Nhân tạo, lớp 23TNT1, Khoa Công nghệ thông tin, Trường Đại học Khoa học tự nhiên - Đại học Quốc gia TP.HCM.}}

\paragraph{}{\textbf{Đồ án được thực hiện bởi nhóm các thành viên:}}

\begin{itemize}
    \item Đinh Đức Anh Khoa (23122001)
    \item Nguyễn Đình Hà Dương (23122002)
    \item Nguyễn Lê Hoàng Trung (23122004)
    \item Đinh Đức Tài (23122013)
\end{itemize}

\section{Phân công nhiệm vụ, đánh giá mức độ hoàn thành}

\textbf{Bảng phân công nhiệm vụ cho từng thành viên:}

\begin{table}[H]
\centering
\begin{tabular}{|c|c|l|c|}
\hline
\textbf{Họ và tên} & \textbf{MSSV} & \multicolumn{1}{c|}{\textbf{Nhiệm vụ}} & \textbf{\begin{tabular}[c]{@{}c@{}}Mức độ\\ hoàn thành\end{tabular}} \\ \hline
\begin{tabular}[c]{@{}c@{}}Đinh Đức \\ Anh Khoa\end{tabular} & 23122001 & \begin{tabular}[c]{@{}l@{}}- Viết báo cáo thuật toán: A*, BFS, so sánh A* và BFS,\\ phát sinh mê cung, hệ thống gợi ý đường đi\\ - Viết code thuật toán A*\\ - Game tester, quay demo\end{tabular} & Tốt (100\%) \\ \hline
\begin{tabular}[c]{@{}c@{}}Nguyễn Đình\\ Hà Dương\end{tabular} & 23122002 & \begin{tabular}[c]{@{}l@{}}- Design hình ảnh menu và game\\ - Viết code giao diện menu, gameplay; xử lý đồ hoạ và xử\\ lý trong gameplay\end{tabular} & Tốt (100\%) \\ \hline
\begin{tabular}[c]{@{}c@{}}Nguyễn Lê\\ Hoàng Trung\end{tabular} & 23122004 & \begin{tabular}[c]{@{}l@{}}- Viết code và báo cáo thuật toán BFS\\ - Viết code giao diện gameplay; xử lí đồ hoạ, dữ liệu của \\ ma trận và xử lí thao tác, chuyển động của nhân vật\end{tabular} & Tốt (100\%) \\ \hline
\begin{tabular}[c]{@{}c@{}}Đinh \\ Đức Tài\end{tabular} & 23122013 & \begin{tabular}[c]{@{}l@{}}- Viết báo cáo. Viết thuật toán mê cung sinh năng lượng.\\ - Viết code giao diện, chức năng menu; code database\\ - Thiết kế cấu trúc phần mềm\end{tabular} & Tốt (100\%) \\ \hline
\end{tabular}
\end{table}

\newpage
\textbf{Tự đánh giá mức độ hoàn thành của từng yêu cầu:}

% Please add the following required packages to your document preamble:
% \usepackage{multirow}
\begin{table}[H]
\centering
\begin{tabular}{|cc|l|c|}
\hline
\multicolumn{2}{|c|}{\textbf{\begin{tabular}[c]{@{}c@{}}Các yêu \\ cầu chính\end{tabular}}} & \multicolumn{1}{c|}{\textbf{Nội dung}} & \textbf{\begin{tabular}[c]{@{}c@{}}Tự đánh giá mức \\ độ hoàn thành\end{tabular}} \\ \hline
\multicolumn{1}{|c|}{\multirow{2}{*}{\begin{tabular}[c]{@{}c@{}}Xử lí\\ tài\\ khoản\end{tabular}}} & \begin{tabular}[c]{@{}c@{}}Đăng \\ nhập\end{tabular} & \begin{tabular}[c]{@{}l@{}}Người dùng nhập tên đăng nhập và mật khẩu để đăng \\ nhập vào game\end{tabular} & Tốt (100\%) \\ \cline{2-4} 
\multicolumn{1}{|c|}{} & \begin{tabular}[c]{@{}c@{}}Đăng \\ kí\end{tabular} & \begin{tabular}[c]{@{}l@{}}Nếu người dùng chưa có tài khoản, phải điền tên và \\ mật khẩu để sử dụng\end{tabular} & Tốt (100\%) \\ \hline
\multicolumn{1}{|c|}{} & \begin{tabular}[c]{@{}c@{}}Chế \\ độ \\ chơi\end{tabular} & \begin{tabular}[c]{@{}l@{}}Lựa chọn chế độ chơi (tự chơi, tự động), độ khó \\ (dễ, trung bình, khó), phát sinh bản đồ (ngẫu nhiên, \\ tự chọn)\end{tabular} & Tốt (100\%) \\ \cline{2-4} 
\multicolumn{1}{|c|}{Menu} & \begin{tabular}[c]{@{}c@{}}Bảng \\ xếp\\ hạng\end{tabular} & \begin{tabular}[c]{@{}l@{}}Sau khi qua mỗi mê cung, người dùng sẽ được lưu \\ thời gian qua màn, số bước đã sử dụng từ đó lưu vào\\ bảng xếp hạng. Bảng xếp hạng được chia theo độ khó.\end{tabular} & Tốt (100\%) \\ \cline{2-4} 
\multicolumn{1}{|c|}{} & \begin{tabular}[c]{@{}c@{}}Thoát \\ game\end{tabular} & Kết thúc trò chơi & Tốt (100\%) \\ \hline
\multicolumn{2}{|c|}{\begin{tabular}[c]{@{}c@{}}Lưu \\ trạng thái \\ người chơi\end{tabular}} & \begin{tabular}[c]{@{}l@{}}Ở chế độ tự chơi, khi người chơi chưa hoàn thành\\ nhưng đã nghỉ giữa chừng thì phải lưu lại được \\ trạng thái và có thể load lại map được khi người \\ dùng quay lại chơi.\end{tabular} & Tốt (100\%) \\ \hline
\multicolumn{2}{|c|}{\textbf{\begin{tabular}[c]{@{}c@{}}Các yêu \\ cầu khác\end{tabular}}} & \multicolumn{1}{c|}{\textbf{Nội dung}} & \textbf{\begin{tabular}[c]{@{}c@{}}Tự đánh giá mức \\ độ hoàn thành\end{tabular}} \\ \hline
\multicolumn{2}{|c|}{\begin{tabular}[c]{@{}c@{}}Gợi ý \\ đường đi\end{tabular}} & \begin{tabular}[c]{@{}l@{}}Ở chế độ tự chơi, khi người dùng nhấn vào nút gợi \\ ý,  hệ thống sẽ hiển thị đường đi từ vị trí hiện tại \\ của nhân vật đến đích.\end{tabular} & Tốt (100\%) \\ \hline
\multicolumn{2}{|c|}{\begin{tabular}[c]{@{}c@{}}Âm nhạc, \\ hình nền\end{tabular}} & Tạo âm nhạc và hình nền cho trò chơi & Tốt (100\%) \\ \hline
\multicolumn{2}{|c|}{\begin{tabular}[c]{@{}c@{}}Lý thuyết \\ trò chơi\end{tabular}} & \begin{tabular}[c]{@{}l@{}}Cho trước số bước đi tối đa của Tâm là H, và phát \\ sinh ngẫu nhiên K viên năng lượng trong bản đồ.\\ Tìm đường đi ngắn nhất đến nhà Gia Huy, biết rằng \\ cứ mỗi bước đi sẽ mất 1 năng lượng và ăn được 1 \\ viên năng lượng sẽ có thêm $V_k$ năng lượng với\\ $V_k$ là giá trị phát sinh ngẫu nhiên thuộc \\ \{1, 2, 3, 4, 5\}. Giá trị của H và K cũng sẽ tương \\ ứng với các mức độ trò chơi.\end{tabular} & Khá (65\%) \\ \hline
\end{tabular}
\end{table}