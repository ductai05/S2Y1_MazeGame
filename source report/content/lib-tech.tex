\newpage
\section{Các thư viện và công nghệ}
\subsection{Python}
\paragraph{Ngôn ngữ lập trình Python} \cite{python}
\paragraph{}{Python là một ngôn ngữ lập trình thông dịch, hướng đối tượng, cấp cao với ngữ nghĩa động. Các cấu trúc dữ liệu tích hợp cấp cao của nó, kết hợp với kiểu động và liên kết động, làm cho Python rất hấp dẫn để phát triển ứng dụng nhanh, cũng như sử dụng như một ngôn ngữ kịch bản hoặc ngôn ngữ kết nối để kết nối các thành phần hiện có với nhau. Cú pháp đơn giản, dễ học của Python nhấn mạnh tính dễ đọc và do đó giảm chi phí bảo trì chương trình. Python hỗ trợ các mô-đun và gói, khuyến khích tính mô-đun của chương trình và tái sử dụng mã nguồn. Trình thông dịch Python và thư viện chuẩn phong phú có sẵn dưới dạng mã nguồn hoặc dạng nhị phân mà không mất phí cho tất cả các nền tảng chính, và có thể được phân phối tự do.}

\paragraph{Phiên bản}
\paragraph{}{Đồ án sử dụng Python phiên bản \textbf{3.11.9}.}

\subsection{Pygame}

\paragraph{Thư viện Pygame} \cite{pygame}
\paragraph{}{Pygame là một bộ mô-đun Python đa nền tảng được thiết kế để viết trò chơi điện tử. Pygame bao gồm đồ họa máy tính và thư viện âm thanh được thiết kế để sử dụng với ngôn ngữ lập trình Python.}

\paragraph{Phiên bản}
\paragraph{}{Đồ án sử dụng Pygame phiên bản \textbf{2.5.2}.}

\subsection{Pygame-menu} 

\paragraph{Thư viện Pygame-menu} \cite{pygame-menu}
\paragraph{}{Pygame-menu là một thư viện python-pygame để tạo menu và giao diện người dùng đồ họa (GUI). Pygame-menu hỗ trợ nhiều widget khác nhau, chẳng hạn như nút bấm, bộ chọn màu, đồng hồ, bộ chọn thả xuống, khung, hình ảnh, nhãn, bộ chọn, bảng, đầu vào văn bản, công tắc màu và nhiều hơn nữa, với nhiều tùy chọn để tùy chỉnh.}

\paragraph{Phiên bản}
\paragraph{}{Đồ án sử dụng Pygame-menu phiên bản \textbf{4.4.3}.}
